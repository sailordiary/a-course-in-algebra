\chapter{代数结构}
初识一个人,你只能记住他们的姓名和相貌.过后再见时,你便会对他们有更深入的了解,甚至和他们成为朋友.

第一章只向你介绍书中涉及到的多数代数结构.想要更深入地理解它们还需一段时间,期间需要阅读和解题.
\section{序言}
如果能精确定义代数的讨论对象,那么它必然是对代数结构的研究.代数结构(algebraic structure)是指定义有运算的集合.集合$M$上的一个运算(operation)是指映射
\[M\times M\to M,\]
也就是一种为$M$中任意两元素指定一个$M$中同类元素的法则.这些元素可以是数,也可以是其他对象.

下面这些数集是代数结构中著名的重要例子.它们定义有加法和乘法运算:

$\bN$,全体自然数的集合;

$\bZ$,全体整数的集合;

$\bZ_+=\bN\cup\{0\}$,全体非负整数的集合;

$\bQ$,全体有理数的集合;

$\bR$,全体实数的集合;

$\bR_+$,全体非负实数的集合.

我们作一个注记,不是每个数集上都定义有加法和乘法运算.例如,负数的集合上就没有定义乘法,因为两个负数之积是正数.在无理数的集合上,乘法和加法都没有定义,因为两个无理数的和与积都可能是有理数.

下面是一些元素非数的代数结构的例子:
\begin{example}
令$M,N,P$为三个集合,并设
\[f:N\to M,\quad g:P\to N\]
为其间的映射.$f$和$g$的\textbf{乘积}(product)或称\textbf{复合}(composite)是映射
\[fg:P\to M,\]
定义为
\[(fg)(a)=f(g(a))\quad\forall a\in P,\]
即$g$和$f$依次作用的结果.特别地,若$M=N=P$,我们就得到全体$M$到自身映射的集合上的一个运算.在一类称为群的代数结构中,这个运算给出了许多重要的例子.例如,根据Euclid几何的公理,平面的两个运动的乘积仍是一个运动.考虑全体这类运动的集合上的乘法运算,就得到了称为平面的运动群的代数结构.
\end{example}
\begin{example}
三维空间中,全体向量连同加法和点积运算就是有两个运算的代数结构的一个例子.但是要注意,内积不是上面定义的这类运算.事实上,它的运算结果和原向量不属于同一个集合.代数中也会考虑内积这样更为一般的运算,但我们暂时不去管它.
\end{example}
上面的例子都是自然的,因为它们都是在真实世界或是数学的内在发展中出现的.但事实上可以考虑任意集合上的任意运算.例如,我们可以把集合$\bZ_+$和这样一个运算一同考虑,对任两个数,它给出十进制表示中相同数码的个数.尽管如此,只有少数代数结构是我们真正感兴趣的.

此外,代数研究者只对可以用其特定运算描述的代数结构和元素感兴趣.同构的概念给出了这一观点的正式表述.
\begin{define}
设$M,N$为两个集合,分别定义了运算$\circ$和$\ast$.称代数结构$(M,\circ )$和$(N,\ast )$是\textbf{同构的},如果存在双射
\[f:M\to N\]
使得
\[f(a\circ b)=f(a)\ast f(b)\]
对一切$a,b\in M$成立.我们把这一事实记作$(M,\circ )\simeq (N,\ast )$.映射$f$称为$(M,\circ )$与$(N,\ast )$的一个\textbf{同构}(isomorphism).
\end{define}
我们类似地定义有两种或更多种运算的代数结构之间的同构.
\begin{example}
映射
\[a\mapsto 2^a\]
是带有加法运算的实数集与带有乘法运算的正实数集之间的同构.事实上,
\[2^{a+b}=2^a2^b.\]
除基数$2$外,我们还可以考虑任意不为$1$的正基数.这表明两个同构的代数结构之间可能存在许多不同的同构.
\end{example}
\begin{example}
设$M$为平面沿某固定直线的平移全体.记实数$a$在$M$中的对应平移为$t_a$,它定义为沿长为$|a|$、方向由$a$的符号给出的向量作平移.容易看出
\[t_{a+b}=t_a\circ t_b,\]
其中$\circ$代表平移变换的乘积(复合).因此,映射$a\to t_a$是代数结构$(\bR , +)$与$(M, \circ)$之间的一个同构.
\end{example}
显然,如果两代数结构同构,则只用运算作出的陈述在一个结构中成立当且仅当它在另一结构中也成立.

例如,集合$M$上的运算$\circ$称为\textbf{交换的}(commutative),若
\[a\circ b=b\circ a\]
对任意$a,b\in M$成立.如果$(M,\circ )$同构于$(N,\ast )$且$M$上的运算$\circ$是交换的,则$N$上的运算$\ast$也是交换的.

所以,研究同构的结构中的哪一个是无关紧要的:它们都是同一对象的不同模型.但是,模型的选取在解决某一特定问题时可能很重要.有时某个特定的模型事实上更有用.例如,具有几何性质的模型可以通过几何方法研究.
\section{Abel群}
\section{环和域}
\section{子群、子环和子域}
\section{复数域}
\section{剩余类环}
\section{向量空间}
\section{代数}
\section{矩阵代数}
