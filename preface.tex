\chapter{译\hspace{2ex}者\hspace{2ex}序}

\chapter{前\hspace{4ex}言}
我写作本书的动机是1992至1994年间于莫斯科独立大学数学学院教授的一门为期两年的代数课程.学生的热情和较少的人数使我可以把讲授的难度保持在高于莫斯科大学数学力学系的通常难度以上,并涉及一些超出一般大学这门课程内容的话题.然而撰写这本书的过程中我利用的是在大学的教学经验,因此最终呈现的版本与我在独立大学讲授的课程只是部分相关.

第一章至第七章及第八章的部分多少对应于莫斯科大学数学力学系第一年代数课的内容.其余章节则覆盖甚至是超出了第二年代数课的内容.这些章节主要是供专攻代数的学生使用.

注意,第七章主要介绍Euclid空间、仿射空间及射影空间中的三种几何.但是,这一章不应视为对几何知识的说明,它介绍的是几何的代数学方法.

前四章中,我尝试介绍得足够具体,使其适合莫斯科大学数学系新生这样的读者阅读.(不过集合和映射的语言从书的最开始便不加解释地使用了.)后续章节中,我便放开手脚跳过了一些可以很容易重新补足的细节,因为在我看来,读者应逐步培养其数学素养.

本书几乎不包括技术上困难的证明.我按照自己对数学的看法,尝试着以概念性的证明取代计算和繁琐的推导.某些读者可能觉得这种风格难以接受,不过对学生来说,花些功夫接受新观念对于解答书中没有涉及的问题来说是值得的.

英译本中,书末的参考文献已加以修订.它自然不是完整的,某种程度上讲甚至有些随意,但我相信读者或许会觉得有帮助.

我要对所有莫斯科大学数学力学系高等代数教研室的现任和前任成员表示感谢,他们帮助我形成了属于自己的代数教学方法.

英译本中修正了数处错印和谬误,并加入了一些解释.
~\\~\\
E. B. 温贝格~(E. B. Vinberg)\hfill 2002年11月于莫斯科